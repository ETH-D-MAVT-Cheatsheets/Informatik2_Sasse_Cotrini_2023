\subsection{Numpy Array operations}
{\centering\underline{\textbf{Common numpy array operations}} \par}
• Return the number of elements:
a = np.arange(10)
a.size() #=10
• Accessing elements (1D array)
a[5] #=5
• Accessing elements (2D array):
A = np.array([[1,2,3],[4,5,6]])
A[1,2] #=A[1][2]=6
A[:,2] #array([3,6])
A[1,:] #array([4,5,6])

{\centering\underline{\textbf{Slicing}} \par}
• Slicing is dependent on the dimensions of the matrix. For 1D arrays:
A = np.arange(10) #[0,1,2,3,4,5,6,7,8,9]
A[2:5:2] #array([2,4])
• For 2D arrays (matrices):
A = np.array([[1,2,3],
	     [4,5,6],
              [7,8,9]])
A[0:2,1:3] #array([2,3],
                   [5,6])

{\centering\underline{\textbf{Statistics}} \par}
a = np.linspace(-4,-2,3) #array([-4,-3,-2])
• Minimum of all elements:
a.min() #-4 
• Maximum of all elements:
a.max() #-2
• Sum of all elements:
a.sum() #-9 
• Average of all elements:
np.mean(a) #-3
• Standard deviation of all elements:
np.std(a) #0.81 

{\centering\underline{\textbf{Mathematical Operations}} \par}
• Generally mathematical operations are carried out element-wise:
A = np.array([[2,3,4],[6,7,6]])
B = np.array([[1,9,1],[2,3,9]])
A+1
#array([[3,4,5],
       [7,8,7]])
A * 2 
#array([[4,6,8],
       [12,14,12]])
A ** 4 
#array([[16,81,256],
       [1296,2401,1296]])
np.sin(A) 
#array([[0.909,0.141,-0.756],
       [-0.279,0.657,-0.279]])
A + B 
#array([[3,12,5],
        [8,10,15]])
A * B 
#array([[2,27,4],
       [12.21,54]])
np.sum(A, axis = 0) 
#= A.sum(axis = 0) -> array([8,10,10])
np.sum(A, axis = 1) 
#= A.sum(axis = 1) -> array([9,19])

• Given that the dimensionality of two matrices is correct, one is able to multiply them using @:
A = np.array([[2,3,4],[6,7,6]])
A @ np.array([[1, 4], [3, 4], [4,6]]) 
#array([[27, 44], 
        [51, 88]])
• Using the dot() function to multiply two matrices:
A.dot(np.array([[1,4],[3,4],[4,6]]) 
#array([[27, 44], 
        [51, 88]])
• Using the dot() function to find the scalar product of two vectors:
a = np.array([1,2,3])
b = np.array([3,4,6])
a.dot(b) #29

{\centering\underline{\textbf{Filtering}} \par}
• Filter a numpy array using the subscript operator:
a = np.arange(7) #array([0,1,2,3,4,5,6])
f = a % 2 == 0
a[f] #array([0,2,4,6])