\subsection{Runtime analysis}
    \subsubsection{upper, lower and tight bound}
    To measure the performance of an algorithm, we use big O notation.
    Let g be the relationship time vs input size for an algorithm.
    
    Upper bound (If g does not grow faster than c*f):
    $g=\mathcal{O}(f)$
    Tight Bound (If g grows about the same as c*f):
    $g= \Theta(f)$
    Lower Bound (If g does not grow slower than c*f):
    $g= \Omega(f)$
    
    Mathematical definitions:
%    O(g)={ f∶ N ⟶ R ┤|  ∃c>0,∃n_0  ∈ N∶ ∀n ≥ n_0 ∶0 ≤f(n)≤c∙g(n)}
%    Θ(g)={ f∶ N ⟶ R ┤|  ∃c>0,∃n_0  ∈ N∶ ∀n ≥ n_0 ∶0 ≤1/c∙g(n)≤f(n)≤c∙g(n)}
%    Ω(g)={ f∶ N ⟶ R ┤|  ∃c>0,∃n_0  ∈ N∶ ∀n ≥ n_0 ∶0 ≤c∙g(n)≤f(n)}
    
    