\subsubsection{Dictionary}
{\centering\underline{\textbf{Initialise Dictionary}} \par}
\begin{lstlisting}
d = {"Lea":22, "Tim":19, "Mortis":69} #key:value
\end{lstlisting}

{\centering\underline{\textbf{Common Dictionary Operations}} \par}
d["Lea"] = 23 #Change item
d["Peter"] = Use 24 #Add item, 24 is an unused key
del d["Mortis"] #delete item
"Tim" in d #Search for key, returns bool
• To access value at a key:
d["Tim"] #has value 19
• Make two lists into one dictionary:
cities = ["Zurich", "Basel", "Bern"] #list 1
zip code = [8000, 4000, 3000] #list 2
d2 = dict(zip(cities,code)) #dictionary D2

2.3.3 Iterating over a Dictionary
• Iterate over the keys of a dictionary:
for key in d.keys():
  print(key) #Lea Tim Mortis
• Iterate over entries of a dictionary:
for item in d.items():
  print(item) #("Lea",22) ("Tim",19) ("Mortis", 69)
• Iterate over entries, with keys and values separated:
for key, value in d.items():
  print(key+" "+value) #Lea 22 Tim 19 Mortis 69
• Iterate over the values of the dictionary:
for value in d.values():
  print(value) #22 19 69




  2.3.4 Dictionary/Set Comprehension
  • Transform a set into a dictionary by applying f(x) and g(x) on every element in the set to obtain key and value, respectively:
  d3 = {f(x):g(x) for x in s} #s being a set
  • Transform a set into a dictionary, only if the element satisfies h(x):
  d4 = {f(x):g(x) for x in s if h(x)}
  • Dictionary comprehension with multiple variables:
  d5 = {f(x):g(y) for x, y in h(z)}
  #h must return a list of tuples, e.g.: zip, d.items. In #the case of d.items, we are applying f(x) on the keys #and g(y) on the values of the dictionary.
  • Example: Multiply the value of every odd key in a dictionary by 2:
  d6 = {k:2*v for k, v in d.items() if k % 2 == 1}
  